\documentclass[12pt]{beamer}
\usepackage{resume/dhuBeamer}

\title{东华大学 dhuBeamer 模板}
\subtitle{beamer 副标题}
\author[Castor]{黄川桂}
\institute{管院 \& 计院}
\date{\today}


\begin{document}
\maketitle


\section{为什么使用 \LaTeX 和 Beamer}

\begin{frame}
    \frametitle{\LaTeX 和 Beamer 简介}

    \LaTeX 是一种用于排版文档的标记语言,广泛用于学术界、出版业和技术文档中。
    它以其专业的排版质量和对数学公式的支持而闻名。
    \LaTeX 使用类似编程的语法,用户通过输入文本和特定命令来描述文档结构和格式,然后通过编译生成最终的文档。

    ~\par

    Beamer 是 \LaTeX 的一个文档类,用于制作演示文稿。它提供了许多功能和样式,使用户能够轻松创建专业和漂亮的演示文稿。
    Beamer 支持幻灯片、动画、表格、数学公式等,同时具有丰富的主题和布局选项,让用户能够自定义演示文稿的外观和感觉。
\end{frame}

\begin{frame}
    \frametitle{\LaTeX 和 PowerPoint 对比}

    \begin{table}[h]
        \centering
        \begin{tabular}{l|l}
            Microsoft\textsuperscript{\textregistered}  PowerPoint & \LaTeX \ Beamer\\
            \midrule
            字处理工具 & 专业排版软件 \\
            容易上手,简单直观 & 容易上手 \\
            所见即所得 & 所见即所想,所想即所得 \\
            高级功能不易掌握 & 进阶难,但一般用不到 \\
            处理长文档需要丰富经验 & 和短文档处理基本无异 \\
            花费大量时间调格式 & 无需担心格式,专心作者内容 \\
            公式排版差强人意 & 尤其擅长公式排版 \\
            二进制格式,兼容性差 & 文本文件,易读、稳定 \\
            付费商业许可 & 自由免费使用 \\
        \end{tabular}
    \end{table}
\end{frame}

\begin{frame}[t]
    \frametitle{公式排版举例}

    \begin{columns}[T,onlytextwidth]
        \begin{column}{0.48\textwidth}
            \begin{block}{行内公式与跨行公式}
                行内公式:$\mathcal{F}(\xi)=\int_{-\infty}^{\infty} f(x)\mathrm{e}^{-\mathrm{j}2\pi
                \xi x}\,\mathrm{d}x$

                跨行公式:
                $$
                \mathcal{F}(\xi)=\int_{-\infty}^{\infty} f(x)\mathrm{e}^{-\mathrm{j}2\pi\xi x}\,\mathrm{d}x
                $$
            \end{block}
        \end{column}
        
        \begin{column}{0.48\textwidth}
            \begin{block}{无编号与有编号公式}
                无编号公式:
                \begin{equation*}
                    \mathcal{F}(\xi)=\int_{-\infty}^{\infty} f(x)\mathrm{e}^{-\mathrm{j}2\pi
                    \xi x}\,\mathrm{d}x
                \end{equation*}

                有编号公式:
                \begin{equation}
                    \mathcal{F}(\xi)=\int_{-\infty}^{\infty} f(x)\mathrm{e}^{-\mathrm{j}2\pi
                    \xi x}\,\mathrm{d}x
                \end{equation}
            \end{block}
        \end{column}
    \end{columns}
\end{frame}

\section{dhuBeamer 模板介绍}

\begin{frame}
    \frametitle{模板参考}

    \begin{itemize}
        \item 模板制作参考
        \begin{itemize}
            \item \href{https://github.com/sjtug/SJTUBeamer}{SJTUBeamer}
            \item \href{https://www.overleaf.com/latex/templates/bei-da-zhong-wen-mo-ban-pku-beamer-template/kfxpbtzrqhrn}{PKU-Beamer-Template}
        \end{itemize}
        \item 模板风格参考
        \begin{itemize}
            \item \href{https://www.dhu.edu.cn/_upload/article/files/d2/8c/2137ec0c44238fd6fbd3ee28ff07/9f9b566a-67f1-4717-991f-477ee5b43acb.zip}{东华大学标准与学术 PPT 模板(2020版)}
        \end{itemize}
        \item 模板元素来源
        \begin{itemize}
            \item \href{https://www.dhu.edu.cn/bsxt/listm.htm}{东华大学标识系统}
            \item 东华大学 LOGO,背景
            \item 东华大学颜色:锦缎红,晨曦红,风帆黄,基石灰
        \end{itemize}
    \end{itemize}

\end{frame}

\begin{frame}
    \frametitle{模板现状}

    本模板由黄川桂于2024年2月发布 v1.0 版本,已在多个平台开源:

    \begin{itemize}
        \item Github: \href{https://github.com/3000ye/dhuBeamer}{dhuBeamer}
        \item Overleaf: 
    \end{itemize}

\end{frame}

\section{dhuBeamer 模板使用说明}


\makebottom
\end{document}

